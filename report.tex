\documentclass[12pt]{article}
\usepackage[english]{babel}
\usepackage{natbib}
\usepackage{url}
\usepackage{amsmath}
\usepackage{graphicx}
\usepackage{parskip}
\usepackage{fancyhdr}
\usepackage{vmargin}
\usepackage{minted}
\graphicspath{{images/}}
\setminted{frame=lines, framesep=2mm, baselinestretch=1.2, fontsize=\footnotesize, linenos, mathescape=true}
\usemintedstyle{vs}
\setmarginsrb{3 cm}{2.5 cm}{3 cm}{2.5 cm}{1 cm}{1.5 cm}{1 cm}{1.5 cm}

\title{Operating Systems Laboratory}                                % Title
\author{Your Id}                               % Author
\date{\today}                                         % Date

\makeatletter
\let\thetitle\@title
\let\theauthor\@author
\let\thedate\@date
\makeatother

\pagestyle{fancy}
\fancyhf{}
\rhead{\theauthor}
\lhead{\thetitle}
\cfoot{\thepage}

\begin{document}

%%%%%%%%%%%%%%%%%%%%%%%%%%%%%%%%%%%%%%%%%%%%%%%%%%%%%%%%%%%%%%%%%%%%%%%%%%%%%%%%%%%%%%%%%

\begin{titlepage}
    \centering
    \vspace*{0.5 cm}
    \includegraphics[scale = 0.75]{logo.png}\\[1.0 cm]  % University Logo
    \textsc{\LARGE Southern University of Science \newline\newline and Technology}\\[2.0 cm]  % University Name
    \textsc{\Large CS303}\\[0.5 cm]               % Course Code
    \rule{\linewidth}{0.2 mm} \\[0.4 cm]
    { \huge \bfseries \thetitle}\\
    \rule{\linewidth}{0.2 mm} \\[1.5 cm]

    \begin{minipage}{0.4\textwidth}

      \begin{centering} \large
        Your Name\\
        Some Colleage\\
        Your Id\\
      \end{centering}
    \end{minipage}\\[2 cm]

\end{titlepage}

%%%%%%%%%%%%%%%%%%%%%%%%%%%%%%%%%%%%%%%%%%%%%%%%%%%%%%%%%%%%%%%%%%%%%%%%%%%%%%%%%%%%%%%%%
\tableofcontents
\pagebreak

%%%%%%%%%%%%%%%%%%%%%%%%%%%%%%%%%%%%%%%%%%%%%%%%%%%%%%%%%%%%%%%%%%%%%%%%%%%%%%%%%%%%%%%%%
\section{Implement a File System}
\subsection{Step 1}
\subsection{Step 2}
\subsection{Code}
This template use minted to provide syntax highlight functions, you need to install it with following commands:
\begin{minted}{bash}
  pip install Pygments
\end{minted}
Here are some examples:
\paragraph{Multi-line of Codes}
\begin{minted}{python}
  import numpy as np

  def incmatrix(genl1,genl2):
      m = len(genl1)
      n = len(genl2)
      M = None #to become the incidence matrix
      VT = np.zeros((n*m,1), int)  #dummy variable

      #compute the bitwise xor matrix
      M1 = bitxormatrix(genl1)
      M2 = np.triu(bitxormatrix(genl2),1)

      # $f = \int_0^{\inf}x^2$
      for i in range(m-1):
          for j in range(i+1, m):
              [r,c] = np.where(M2 == M1[i,j])
              for k in range(len(r)):
                  VT[(i)*n + r[k]] = 1;
                  VT[(i)*n + c[k]] = 1;
                  VT[(j)*n + r[k]] = 1;
                  VT[(j)*n + c[k]] = 1;

                  if M is None:
                      M = np.copy(VT)
                  else:
                      M = np.concatenate((M, VT), 1)

                  VT = np.zeros((n*m,1), int)

      return M
\end{minted}
\section{Implement a Cache}
\newpage
\bibliographystyle{plain}
\bibliography{biblist}

\end{document}
